\section{Geschichtlicher Hintergrund und Begriffe}
\subsection{Die Geschichte}
Der russische Mathematiker L. V. Kantorovich veröffentlichte 1939 das Buch "Mathematische Methode in der Organisation und Planung der Produktion", welche mithilfe von mathematischen Modellen die Kosteneffizienz in der Produktion eines Betriebes steigern soll. Die Arbeit Kantorovichs ist insofern bedeutend, da er als erstes erkannte, dass bestimmte Arten der Produktion, definierte mathematische Strukturen besitzen, die numerisch gelöst werden können. Allerdings wurde zu jener Zeit die Bedeutung dieser Arbeit nicht in vollem Umfang erkannt. Mitte der 1940-er Jahre wurde Georg Dantzig bewusst, dass sich viele praktische ökonomische Beschränkungen bei der Modellierung von Planungsaufgaben durch lineare Ungleichungen beschreiben lassen. Er ersetzte erstmals bewusst die bis dahin geltenden Regeln zur Lösung von Planungsproblemen durch eine Zielfunktion und Nebenbedingungen, in Form von Gleichungen und Ungleichungen. Dadurch entstand eine klare Trennung zwischen dem Ziel der Optimierung, der zulässigen Lösungsmenge und den Mitteln zur Lösung des Problems. 1950 wurde der Simplex-Algorithmus zum ersten Mal mit Computern zur Lösung des Transportproblems genutzt. Die Nutzung von Computern in Betrieben wuchs immens. Lineare Optimierung wurde für die Industrie zwischen den 1955-1960 wichtiger und sie wurde beispielsweise vom Management als Verfahren zur Optimierung der Effizienz des Betriebes genutzt. Die Öl-Industrie und die Landwirtschaft waren Wirtschaftszweige, welche die Lineare Optimierung besonders effizient anwenden konnten. Seit 1957 wuchsen die Anwendungsbereiche immer weiter. 
\\
\subsection{Wichtige Begriffe/Das Verfahren}

Damit der Simplex-Algorithmus angewendet werden kann, müssen Nebenbedingungen und eine Zielfunktion bestimmt werden. Daraufhin muss entschieden werden, ob die Zielfunktion maximiert oder minimiert wird. Danach wird das Problem in eine Art Matrixform – das Simplex-Tableau – übertragen. Mithilfe von sogenannten Schlupfvariablen s1 und s2 wird es möglich, dass aus den beiden Ungleichungen zwei Gleichungen gebildet werden. Es wird jeweils nur eine Schlupfvariable pro Zeile genutzt, sodass sie in der Simplex Tabelle eine Einheitsmatrix bilden. Die Zielfunktion des Algorithmus wird vorzugsweise an der untersten Stelle der Tabelle eingetragen. Neben den Schlupfvariablen gibt es noch Basisvariablen. Als Basisvariable werden die x- Werte in den Nebenbedingungen bezeichnet. Die Werte, die in der Zielfunktion anstelle der Schlupfvariablen stehen, werden als Schattenpreis bezeichnet. Das bedeutet, wenn die Menge für x1 um 1 erhöht wird, wirkt sich das auf den Gesamtwert um den Wert bei dem Schattenpreis aus. Die Pivotspalte sowie die Pivotzeile sind jede Zeile und Spalten, die beim Algorithmus ausgewählt werden, um das Pivotelement zu bestimmen. Im weiteren Verlauf wird anhand eines Beispiels erläutert, wie die Pivotzeile und die Pivotspalte sowie dass Element ausgewählt werden müssen.





