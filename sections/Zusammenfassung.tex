\section{Zusammenfassung }
\subsection{Anmerkungen}
Anzumerken ist noch, dass man einige Probleme mit dem Simplex-Verfahren lösen kann, sich jedoch nicht auf dieses Ergebnis blind verlassen sollte, da die Nebenbedingungen nicht immer ideal und statisch sind und man hier auch nur Idealfälle ausrechnet.  \\
Bei einer kleineren Menge an Nebenbedingungen, wie in diesem Problem, bei welchem davon ausgegangen wird, dass niemand Krank wird, an bestimmten Tagen Termine hat und immer die gleiche Zeit auf der Arbeit verbringt, kann man das Ergebnis als grobe Richtlinie verwenden, jedoch wird dieses Ergebnis immer unrealistischer, je realistischer die Bedingungen werden.
\subsection{Zusammenfassung: }
Nachdem das Simplex-Verfahren zuerst keine Beachtung fand, konnte es  sich sukzessive einen Weg in die Wirtschaft bahnen. 
Bei der Bearbeitung eines linearen Optimierungsproblems mittels Simplex-Verfahren kann es jedoch auch zu Sonderfällen kommen, die nicht optimal gelöst werden können.
Um das Simplex-Verfahren rechnergestützt anzuwenden gibt es einige Programme, deren Einsatz eine sehr schnelle Berechnung zulassen. Es kann zudem gleich eine Sensitivitätsanalyse stattfinden. 
Die Sensitivitätsanalyse ist ein zuverlässiges Mittel, um die Stabilität und gleichzeitig Flexibilität der Zielfunktion zu bestimmen. Hierbei gilt es jedoch zu beachten, dass, wie auch das Simplex-Verfahren selbst, es sich um eine rein mathematische Optimallösung handelt und es somit nur einen Teil zur realen Problemlösung beitragen kann.
