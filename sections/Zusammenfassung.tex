\section{Zusammenfassung }

Zusammengefasst kann man sagen, dass man einige Probleme mit dem Simplex-verfahren lösen kann, sich jedoch nicht auf dieses verlassen sollte, da die Nebenbedingungen nicht immer Ideal und statisch sind und man hier auch nur Idealfälle ausrechnet. Das Problem hierbei ist, dass, je dynamischer die Nebenbedingungen werden, desto höher wird die benötigte Rechenleistung, sodass in großen Firmen oder bei anderen Problemen sich das ausrechnen eines theoretischen Idealfalls nicht mehr lohnt, da von vornherein klar ist, dass dieser höchstwahrscheinlich nicht eingehalten werden kann. \\
Bei einer kleineren menge an Nebenbedingungen wie in diesem Problem, bei welchem davon ausgegangen wird, dass niemand Krank wird, an bestimmten Tagen Termine hat und immer die gleiche Zeit auf der Arbeit verbringt, kann man das Ergebnis als grobe Richtlinie verwenden, jedoch wird dieses Ergebnis immer unrealistischer, je realistischer die Bedingungen werden.

\subsection{Kurs Reflexion}

Der Gruppe hat das Rechnen mit dem Simplex-verfahren Spaß gemacht und es war äußerst interessant ${\displaystyle \mathrm {L\!\!^{{}_{A}}\!\!\!\!\!\;\;T\!_{\displaystyle E}\!X} }$ und die Simplex-Funktionen von Excel kennen zu lernen sowie die Idealfälle graphisch darzustellen. Das Thema ist sehr umfangreich und komplex und man kann, wenn man möchte Stundenlang Werte hin und her schieben und vergleichen.

{\bf Note:} Für spätere Notizen