\section{Beispiel}
Um das Verfahren zu Veranschaulichen, haben wir ein kleines Beispielproblem zusammengebaut, welches wir mithilfe des Simplex Verfahrens lösen.\\~\\
Die Nebenbedingungen für das Problem lauten:\\~\\
\(4x_1+x_2\le3\)\\
\(2x_1+4x_2\le8\)\\~\\
Unter diesen Bedingungen soll die Zielfunktion\\~\\
\(Z = 2x_1+ 3x_2\)\\~\\
maximiert werden\\
\\
\begin{center}
    
\textbf{Schritt 1: \\Ungleichungen zu Gleichungen machen, mithilfe von Schlupf-variablen }\end{center}

\(x_1+x_2+s_1=3\)\\
\(2x_1+6x_2+s_2=8\)\\~\\~\\~\\
\begin{center}
%ab hier sieht man manchmal ein $ vor und hinter einigen Zahlen, dieses startet den "math mode" sodass die Zahl hinter dem x klein und unten ist
\textbf{Schritt 2:\\In Simplex Tableau eintragen}\\\end{center}
\begin{table}[h]
\begin{tabular}{|l|l|l|l|l|l|}
\hline
\rowcolor[HTML]{C0C0C0} 
Zeile                     & $x_1$ & $x_2$ & $s_1$ & $s_2$ & Rechte Spalte \\ \hline
\cellcolor[HTML]{C0C0C0}1 & 1  & 1  & 1  & 0  & 3             \\ \hline
\cellcolor[HTML]{C0C0C0}2 & 2  & 4  & 0  & 1  & 8             \\ \hline
\cellcolor[HTML]{C0C0C0}Z & 2  & 3  & 0  & 0  & 0             \\ \hline
\end{tabular}
\end{table}

\begin{center}\textbf{Schritt 3: \\Pivotelement bestimmen}\end{center}
Hierzu muss der größte Wert der untersten (3.) Zeile rausgesucht werden. Die Spalte, in der sich dieser Wert befindet ist die Pivotspalte, deren Werte dann mit der Rechten Spalte dividiert werden. Die Zeile, wo hierbei der kleinste Wert raus kommt ist die Pivotzeile. Da, wo Pivotspalte und Pivotzeile sich treffen, befindet sich das Pivotelement.\\
\begin{table}[h]
\begin{tabular}{|
>{\columncolor[HTML]{C0C0C0}}l |
>{\columncolor[HTML]{FFFFFF}}l |
>{\columncolor[HTML]{CBCEFB}}l |
>{\columncolor[HTML]{FFFFFF}}l |
>{\columncolor[HTML]{FFFFFF}}l |
>{\columncolor[HTML]{FFFFFF}}l |l|}
\hline
Zeile & \cellcolor[HTML]{C0C0C0}$x_1$& \cellcolor[HTML]{C0C0C0}$x_2$ & \cellcolor[HTML]{C0C0C0}$s_1$ & \cellcolor[HTML]{C0C0C0}$s_2$ & \cellcolor[HTML]{C0C0C0}Rechte Spalte & \cellcolor[HTML]{C0C0C0}RS/PS \\ \hline
1     & 1                          & 1                          & 1                          & 0                          & 7                                     & 3                             \\ \hline
2     & \cellcolor[HTML]{CBCEFB}2  & \cellcolor[HTML]{FD6864}4  & \cellcolor[HTML]{CBCEFB}0  & \cellcolor[HTML]{CBCEFB}1  & \cellcolor[HTML]{CBCEFB}8             & \cellcolor[HTML]{CBCEFB}2     \\ \hline
Z     & 2                          & 3                          & 0                          & 0                          & 0                                     &                               \\ \hline
\end{tabular}
\end{table}

\begin{center}\textbf{Schritt 4: \\Pivotelement zu einer 1 umrechnen}\\\end{center}

\begin{table}[h]
\begin{tabular}{|l|l|l|l|l|l|l|}
\hline
\rowcolor[HTML]{C0C0C0} 
Zeile                     & $x_1$  & $x_2$                        & $s_1$ & $s_2$  & Rechte Spalte & Rechnung                      \\ \hline
\cellcolor[HTML]{C0C0C0}1 & 1   & \cellcolor[HTML]{FFFFFF}1 & 1  & 0   & 3             & -(Zeile 2)                    \\ \hline
\rowcolor[HTML]{FFFFFF} 
\cellcolor[HTML]{C0C0C0}2 & 1/2 & \cellcolor[HTML]{68CBD0}1 & 0  & 1/4 & 2             & (vorab wurde durch 4 geteilt) \\ \hline
\cellcolor[HTML]{C0C0C0}Z & 2   & \cellcolor[HTML]{FFFFFF}3 & 0  & 0   & 0             & -3*(Zeile 2)                  \\ \hline
\end{tabular}
\end{table}
.\\
\begin{center}
\textbf{Schritt 5:\\Werte in Pivotspalte auf 0 bringen }\end{center}
\begin{table}[!ht]
\begin{tabular}{|l|l|l|l|l|l|}
\hline
\rowcolor[HTML]{C0C0C0} 
Zeile                     & $x_1$  & $x_2$ & $s_1$ & $s_2$   & Rechte Spalte \\ \hline
\rowcolor[HTML]{FFFFFF} 
\cellcolor[HTML]{C0C0C0}1 & $1/2$ & 0  & 1  & $-1/4$ & 1             \\ \hline
\rowcolor[HTML]{FFFFFF} 
\cellcolor[HTML]{C0C0C0}2 & $1/2$ & 1  & 0  & $1/4$  & 2             \\ \hline
\rowcolor[HTML]{FFFFFF} 
\cellcolor[HTML]{C0C0C0}Z & $1/2$ &    & 0  & $-3/4$ & -6            \\ \hline
\end{tabular}
\end{table}
\begin{center}\textbf{Schritt 6:\\ Wiederholen}\end{center}
Da wir in unserer Zielfunktion nur negative Werte haben wollen, müssen wir den Vorgang ab 3. so oft wiederholen, bis dies der Fall ist.\\
\begin{table}[!ht]
\begin{tabular}{|l|l|l|l|l|l|l|}
\hline
\rowcolor[HTML]{C0C0C0} 
Zeile                     & $x_1$                          & $x_2$                        & $s_1$                        & $s_2$                          & Rechte Spalte             & RS/PS \\ \hline
\rowcolor[HTML]{9698ED} 
\cellcolor[HTML]{C0C0C0}1 & \cellcolor[HTML]{CE6301}1/2 & 0                         & 1                         & $-1/4$                        & 1                         & 2     \\ \hline
\cellcolor[HTML]{C0C0C0}2 & \cellcolor[HTML]{68CBD0}$1/2$ & \cellcolor[HTML]{FFFFFF}1 & \cellcolor[HTML]{FFFFFF}0 & \cellcolor[HTML]{FFFFFF}$1/4$ & \cellcolor[HTML]{FFFFFF}2 & 4     \\ \hline
\cellcolor[HTML]{C0C0C0}Z & \cellcolor[HTML]{68CBD0}$1/2$ & \cellcolor[HTML]{FFFFFF}0 & 0                         & $-3/4$                       & -6                        &       \\ \hline
\end{tabular}
\end{table}

\begin{table}[!ht]
\begin{tabular}{|l|l|l|l|l|l|l|}
\hline
\rowcolor[HTML]{C0C0C0} 
Zeile                     & $x_1$  & $x_2$ & $s_1$ & $s_2$  & Rechte Spalte & Rechnung                   \\ \hline
\rowcolor[HTML]{FFFFFF} 
\cellcolor[HTML]{C0C0C0}1 & 1   & 0  & 2  & $-1/2$ & 2             & (vorab wurde *2 gerechnet) \\ \hline
\rowcolor[HTML]{FFFFFF} 
\cellcolor[HTML]{C0C0C0}2 & $1/2$ & 1  & 0  & $1/4$  & 2             & -0.5*(Zeile 1)             \\ \hline
\rowcolor[HTML]{FFFFFF} 
\cellcolor[HTML]{C0C0C0}Z & $1/2$ & 0  & 0  & $-3/4$ & -6            & -0.5*(Zeile 1)             \\ \hline
\end{tabular}
\end{table}

\begin{table}[!h]
\begin{tabular}{|l|l|l|l|l|l|}
\hline
\rowcolor[HTML]{C0C0C0} 
Zeile                     & $x_1$ & $x_2$ & $s_1$ & $s_2$   & Rechte Spalte \\ \hline
\rowcolor[HTML]{FFFFFF} 
\cellcolor[HTML]{C0C0C0}1 & 1  & 0  & 0  & 0    & 2             \\ \hline
\rowcolor[HTML]{FFFFFF} 
\cellcolor[HTML]{C0C0C0}2 & 0  & 2  & -2 & $1/2$  & 2             \\ \hline
\rowcolor[HTML]{FFFFFF} 
\cellcolor[HTML]{C0C0C0}Z & 0  & 0  & -2 & $-3/4$ & -7            \\ \hline
\end{tabular}
\end{table}
\begin{center}\textbf{Schritt 7: \\Werte ablesen }\\\end{center}
Da in der untersten Zeile alle Werte negativ sind, ist der Prozess beendet und Z kann berechnet, bzw. abgelesen werden.\\
Aus der Tabelle kann man entnehmen, dass \(x_1=2\) und \( x_2=1 \)optimal sind. Der maximale Zielwert beträgt dabei 7. 