\section{Methoden}

Die drei Simplex Methoden sind der primale Simplex, der duale Simplex und die Big M Methode. Zum optimieren unseres Problems haben wir [später einfügen] verwendet.\\\\
Der duale Simplexalgorithmus wird angewendet, wenn die Werte der rechten Seite der Nebenbedingungen negativ sind. Der primale Simplexalgorithmus wird angewendet, wenn alle Werte der rechten Seite positiv sind. Der duale Simplexalgorithmus führt zu einer zulässigen Ausgangslösung, der primale Simplexalgorithmus zu einer optimalen Lösung. Nach Anwendung des dualen Simplexalgorithmus' kann der primale Simplexalgorithmus angewendet werden, um eine optimale Lösung zu erhalten. 
\\
Die Big-M-Methode ist eine weitere Variante des Simplex Verfahrens. Mit ihr kann eine optimale Lösung gefunden werden, ohne eine Einheitsmatrix mit Schlupfvariablen bilden zu müssen.
\\
Allerdings funktioniert das nur, wenn:\\
\\

