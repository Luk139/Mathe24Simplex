\section{Aufgabenstellung}
Nachdem wir an einem einfachen Beispiel den Simplex-Algorithmus und seine Sonderfälle dargestellt haben, wenden wir uns der Aufgabenstellung zu.
Hierbei geht es um ein Minimierungsproblem mit mehreren Nebenbedingungen. 
\subsection{Ziele des Projekts}
Die Aufgabe ist das Optimieren einer Personaleinsatzplanung.\\
Ein metallverarbeitendes Unternehmen in Oberberg benötigt für die Produktion -rund um die Uhr- Arbeiter.\\
Jeder Arbeiter arbeitet 4 Stunden, macht eine gewerkschaftliche Stunde Pause und arbeitet danach weitere 4 Stunden.\\
Die Arbeitszeit kann zu jeder beliebigen Stunde beginnen und wird mit 10€/Stunde bezahlt, es gibt einen Zuschlag von 2,50€/Stunde für Schichten in der Nacht von 10 Uhr abends bis 6 Uhr morgens.\\
\\
Die hier gegebenen Nebenbedingungen sind der Bedarf an Arbeitern zu bestimmten Uhrzeiten.\\
\\\\
12 Uhr nachts bis 6 Uhr morgens mindestens 2\\
6 Uhr morgens bis 10 Uhr morgens mindestens 8\\
10 Uhr morgens bis 12 Uhr mittags mindestens 4\\
12 Uhr nachmittags bis 4 Uhr nachmittags mindestens 3\\
4 Uhr nachmittags bis 6 Uhr nachmittags mindestens 6\\
6 Uhr nachmittags bis 10 Uhr abends mindestens 5\\
10 Uhr abends bis 12 Uhr nachts mindestens 3\\
\\\\
Die Zielfunktion ergibt sich aus dem Lohn, der für die jeweilige Schicht gezahlt werden muss (Koeffizient) und der Anzahl der Arbeiter, die für diese Schicht eingestellt werden müssen $(x_1-x{24})$.\\
Daraus resultiert folgende Zielfunktion: $z = 90*x_1 + 90*x_2 + 87.5*x_3 + 85*x_4 + 82.5*x_5 + 80*x_6 + 80*x_7 + 80*x_8 + 80*x_9 + 80*x_{10} + 80*x_{11} + 80*x_{12} + 80*x_{13} + 82.5*x_{14} + 85*x_{15} + 87.5*x_{16} + 90*x_{17} + 90*x_{18} + 92.5*x_{19} + 95*x_{20} + 97.5*x_{21} + 97.5*x_{22} + 95*x_{23} + 92.5*x_{24}$