\section{Aufgabenstellung}
Da wir nun wissen wie man das Simplexverfahren anhand eines einfachen Problems anwendet und auf welche Sonderfälle man achten muss, setzen wir uns nun an ein größeres und komplizierteres Problem. Hier geht es um ein Minimierungsproblem mit mehreren Nebenbedingungen. 
\subsection{Ziele des Projekts}
Die Aufgabe ist das Optimieren des Problems einer Personaleinsatzplanung.\\
Ein metallverarbeitendes Unternehmen im Oberberg benötigt für die Produktion -rund um die Uhr- Arbeiter.\\
Jeder Arbeiter arbeitet 4 Stunden, macht eine gewerkschaftliche Stunde Pause und arbeitet danach weitere 4 Stunden.\\
Die Arbeitszeit ann zu jeder beliebigen Stunde beginnen und wird mit 10€/Stunde bezahlt, es gibt einen Zuschlag von 2,50€/Stunde für Schichten in der Nacht von 10 Uhr abends bis 6 Uhr morgens.\\
Die hier gegebenen Nebenbedingung ist der Bedarf an Arbeitern zu bestimmten Urzeiten.\\
12 Uhr nachts bis 6 Uhr morgens mindestens 2\\
6 Uhr morgens bis 10 Uhr morgens mindestens 8\\
10 Uhr morgens bis 12 Uhr mittags mindestens 4\\
12 Uhr nachmittags bis 4 Uhr nachmittags mindestens 3\\
4 Uhr nachmittags bis 6 Uhr nachmittags mindestens 6\\
6 Uhr nachmittags bis 10 Uhr abends mindestens 5\\
10 Uhr abends bis 12 Uhr nachts mindestens 3\\
\\\\
Unter Berücksichtigung dieser Bedingungen, wie soll der Personaleinsatz geplant werden, wenn das Ziel heißt, die Anzahl der Arbeiter und damit die Kosten zu minimieren?

